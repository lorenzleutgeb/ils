\documentclass[xcolor={table,usenames,dvipsnames}]{beamer}

% Suggested improvements:
%  - Actually apply AllMinAs to the example in a step-by-step manner.
%  - Remove references, they are too noisy.
%  - Expand on Complexity.
%  - How is this useful to the person working with the ontology? (maybe mention MaxAs)
%  - Use overlays consistently throughout the presentation.
%  - Maybe explain the algorithms using text and bullet lists with overlays.

\usetheme[sectionpage=progressbar,subsectionpage=progressbar,block=fill]{metropolis}

\usepackage{amsthm}
\usepackage{xspace}
%\usepackage[scale=1.5]{ccicons}
\usepackage{pdfpages}

\usepackage{hyperref}

\usepackage{amsmath,amssymb} % For \flalign, \top, ...
\usepackage[algoruled,linesnumbered]{algorithm2e}

\usepackage{nicefrac,xspace,csquotes}

% If we want article-style math even in the presentation.
%\usefonttheme[onlymath]{serif}

\title{Empirical Evaluation of SAT Provers}
\author{Filippo De Bortoli \texorpdfstring{\newline \texttt{\tiny filippo.debortoli@stud-inf.unibz.it} \newline }{<filippo.debortoli@stud-inf.unibz.it>}%
\and Aneta Koleva \texorpdfstring{\newline \texttt{\tiny aneta.koleva@stud-inf.unibz.it} \newline}{<aneta.koleva@stud-inf.unibz.it>}%
\and Lorenz Leutgeb \texorpdfstring{\newline \texttt{\tiny lorenz.leutgeb@stud-inf.unibz.it} \newline}{<lorenz.leutgeb@stud-inf.unibz.it>}}

\institute{Free University of Bozen-Bolzano}

% \\[10mm] \begin{center}\ccbysa \\[1mm] \tiny{This work is licensed under \href{http://creativecommons.org/licenses/by-sa/4.0/}{Creative Commons BY-SA 4.0}.}\end{center}

\date{2018-03-14}

\begin{document}

\begin{frame}[plain]
\maketitle
\end{frame}

\begin{frame}[plain]
This work is reproduction of and based on:\\[3mm]

{\large\bfseries Generating hard satisfiability problems}\\[1mm]
B.\ Selman, D.\ G.\ Mitchell, H.\ J.\ Levesque\\[2mm]
Artificial Intelligence, Volume 81, Issue 1, 1996, Pages 17-29\\[1mm]
DOI \href{http://dx.doi.org/10.1016/0004-3702(95)00045-3}{\texttt{10.1016/0004-3702(95)00045-3}}\\[3mm]
{\small \color{gray}{(detailed reference in the end)}}
\end{frame}

\begin{frame}{Agenda}
\tableofcontents
\end{frame}

\begin{frame}{Hypotheses}
\begin{description}
	\item[2SAT]{Phase transition does not manifest as harder instances.}
    \item[3SAT]{Results can be reproduced.}
    \item[5SAT]{Phase transitions behaves similar to 3SAT.}
\end{description}

\alert{Note:} We restrict ourselves to the fixed clause length formulae and omit the fixed density generation scheme.
\end{frame}

\begin{frame}{Measuring \enquote{Hardness}}
\begin{description}
    \item[Number of Decisions]{Good discrete measure of hardness, relates to \alert{Splitting Rule}  in Davis-Putnam Procedure.}
	\item[Execution Time]{Overall indicator, more fragile.}
\end{description}
\end{frame}

\section{$k = 2$}

{
\setbeamercolor{background canvas}{bg=}
\includepdf{satisfiability-2.pdf}
\includepdf{combined-2.pdf}
}

\section{$k = 3$}

{
\setbeamercolor{background canvas}{bg=}
\includepdf{satisfiability-3.pdf}
\includepdf{combined-3.pdf}
}

\section{$k = 5$}

{
\setbeamercolor{background canvas}{bg=}
\includepdf{satisfiability-5.pdf}
\includepdf{combined-5.pdf}
}

\begin{frame}{Recap}
\tableofcontents
\end{frame}

\begin{frame}[standout]
Questions, please!
\end{frame}

\end{document}
