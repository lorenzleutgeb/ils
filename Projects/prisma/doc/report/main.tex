\documentclass{llncs}

\usepackage[utf8]{inputenc}
\usepackage[T1]{fontenc}
\usepackage[english]{babel}

\usepackage{hyperref}
\usepackage{bookmark}
\usepackage{csquotes}

\title{Prisma: A SAT Solver Front-End}
\subtitle{Project Report}
\author{Team White\\[2mm]Filippo~De~Bortoli \and Aneta~Koleva \and Lorenz~Leutgeb}
\institute{Free University of Bozen-Bolzano\\[3mm] \texttt{\{\href{mailto:filippo.debortoli@stud-inf.unibz.it}{filippo.debortoli},\href{mailto:aneta.koleva@stud-inf.unibz.it}{aneta.koleva},\href{mailto:lorenz.leutgeb@stud-inf.unibz.it}{lorenz.leutgeb}\}\newline @stud-inf.unibz.it}}

\begin{document}

  \maketitle

  \begin{abstract}
We present an implementation of an optimizing compiler that translates formulae in a high-level language of logical expressions (standard Boolean connectives, quantification of integer expressions, symbolic terms and predicates over finite domains) into ground formulae in CNF. It interfaces with a SAT solver to compute models where propositions correspond to atoms in the high-level input. The resulting Java program is called \enquote{Prisma}.
  \end{abstract}

  \section{Introduction}

% TODO
%
% - describe the task that has been solved
% - outline structure of the report 
  
		
			\begin{tikzpicture}[
			node distance=1cm and 5mm,
			every node/.style={font=\sffamily},
			title/.style={font=\color{black!50}\sffamily},
			typetag/.style={rectangle, draw=black!50, font=\sffamily, anchor=west, text height=3mm, align=center}
			]
			
			\node (lp) at (0, -1cm) {};
			\node (as) at (0, -3.2cm) {};
			
			\node (gl) at (3.6cm, 0) [align=center, text width=2.1cm, title] { \large Python };
			
			\node (par) [below=of gl.west, text width=22mm, typetag] { Simulator };
			\node (gro) [below=1.7cm of par.west, text width=22mm, typetag] { Agent Host };
			
			\node (g) [draw=gray!10, line width=1mm, inner sep=5pt, fit={(gl) (par) (gro)}] {};
			
			\node (sl) at (9.6cm, -1.7cm) [align=center, text width=2.7cm, title] { \large DLV };
			
			\node (ngs) [below=of sl.west, text width=3cm, typetag] { Agent };
			
			\node (s) [draw=gray!10, line width=1mm, inner sep=5pt, fit={(sl) (ngs)}] {};
			
			\draw [->, draw=black!80] (gro) to [out=340, in=200] node [midway, below] {State as Facts} (ngs);
			\draw [->, draw=black!80] (ngs) to [out=160, in=20] node [midway, above] {Answer Set} (gro);
			
			\draw [->, draw=black!80] (par) to [out=280, in=80] node [midway, right] {Perceptions} (gro);
			\draw [<-, draw=black!80] (par) to [out=260, in=100] node [midway, left] {Actions} (gro);
			
			\draw [->, draw=black!80] (lp) -- (par) node [midway, above] {World};
			\end{tikzpicture}
	
  \section{Algorithms and Design Choices}

% TODO
%
% - Describe different versions of the front-end
% - For each relevant versions, describe used algorithms
% - e.g. version with Tseitin, version with interleaving quants, version with unit propagation, ...
% - desiderata: pseudocode for relevant algorithms

  \section{Evaluation}

% TODO
%
% - describe the structure of the test suite
% - is it necessary to describe machine specs for tests run?
%    I don't think so, we are not checking CPU time or machine-related data. ~Filippo
% - describe collected data (variables, clauses, models, ...)
% - desiderata: plot describing evolution of different versions of the front-end
% - arrogant: comparisons with data from other teams on same test suite.
  \section{Conclusions}

% TODO
%
% - briefly summarise report
% - Highlight possible additional features

\section{Running Prisma}

We first explain the prerequisites to running Prisma and how to obtain an executable version in the following. Then, commandline arguments are described.

\subsection{Prerequisites and Obtaining an Executable Version}
\begin{enumerate}
	\item{Ensure that a Java Runtime Environment in version 8 or higher is installed on your machine. To do this, you may run the following command:
	\begin{verbatim}
	$ java -version
	\end{verbatim}
	This should print a version number. For example: \texttt{java~version~"10"} (Java 10) or \texttt{java~version~"1.8.0\_162"} (Java 8u162), both of which are suitable to run Prisma. In case the command is not found, please first install a Java Runtime Environment (there are numerous tutorials and packages for many Linux distributions available online).
	}
	\item{Download a packaged version of Prisma from
	\begin{center}
	\url{https://lorenz.leutgeb.xyz/tmp/prisma.jar}
	\end{center} and let the location of the file be denoted as \texttt{\$PRISMA} in the following.}
	\item{You may run Prisma as follows:
	\begin{verbatim}
	$ java -jar $PRISMA
	\end{verbatim}
	}

\subsection{Commandline Arguments}

Prisma offers four different modes of operation, which roughly correspond to the type of output it is expected to emit:

\begin{description}
	\item[\texttt{CNF}]{Emit a formula that is equisatisfiable with the input formula in the same language as the input languge.}
	\item[\texttt{DIMACS}]{Emit a SAT instance in DIMACS format, including comments that encode which propositional variable maps to which high-level atom.}
	\item[\texttt{SOLVE}]{Print models of the input formula, or \texttt{UNSATISFIABLE} if there are none.}
	\item[\texttt{REPL} \textnormal{(default)}]{Starts a Read Evaluate Print Loop (REPL) for more fine-grained interaction scenarios.}
\end{description}

\end{enumerate}
\end{document}