\documentclass{llncs}

\usepackage[utf8]{inputenc}
\usepackage[T1]{fontenc}
\usepackage[english]{babel}

\usepackage{hyperref}
\usepackage{bookmark}
\usepackage{csquotes}

\title{??: A Player Agent for Wumpus}
\subtitle{Project Report}
\author{Team White\\[2mm]Filippo~De~Bortoli \and Aneta~Koleva \and Lorenz~Leutgeb}
\institute{Free University of Bozen-Bolzano\\[3mm] \texttt{\{\href{mailto:filippo.debortoli@stud-inf.unibz.it}{filippo.debortoli},\href{mailto:aneta.koleva@stud-inf.unibz.it}{aneta.koleva},\href{mailto:lorenz.leutgeb@stud-inf.unibz.it}{lorenz.leutgeb}\}\newline @stud-inf.unibz.it}}

\begin{document}

\maketitle

\begin{abstract}
  The assigned task is to develop an intelligent agent that plays the \emph{Wumpus} game, by using a logic-based approach in its implementation.
  In order to complete a run of the game, this agent has to be able to face several challenges, like the incompleteness of the available information about the state of the world or the search for the best strategy to employ.
  The chosen approach is to develop a hybrid agent that relies on an ASP core to actuate its strategy and on graph-theory techniques to obtain additional insights on how to proceed in the exploration of a dungeon.
  To assess the performance of our agent, we implemented an omniscient agent that obtains an optimal score for a given dungeon and we ranked our agent against it.
  In this report, we introduce our solution to this task, by detailing the architecture of the agent and describing the chosen strategy, the heuristics and the obtained results.
\end{abstract}

%\section{Introduction}

% TODO
%
% - describe the task that has been solved
% - outline structure of the report 
%\section{Installation and Usage}
\label{sec:usage}

\subsection{An example}
%\input{sections/strategy}
%\section{Evaluation}

% TODO
%
% - describe the structure of the test suite
% - is it necessary to describe machine specs for tests run?
%    I don't think so, we are not checking CPU time or machine-related data. ~Filippo
% - describe collected data (variables, clauses, models, ...)
% - desiderata: plot describing evolution of different versions of the front-end
% - arrogant: comparisons with data from other teams on same test suite.

\end{document}
