\documentclass[smaller,dvipsnames,ratio=169]{beamer}

\usetheme[numbering=fraction,%
          block=fill,%
          sectionpage=progressbar,%
          subsectionpage=progressbar,%
  ]{metropolis} % Use metropolis theme
\setbeamercovered{invisible}

\usepackage[utf8]{inputenc}
\usepackage{xcolor}
\usepackage{xspace}
\usepackage{booktabs}
\usepackage{amssymb}
\usepackage{tikz}
  \usetikzlibrary{arrows} % required in the preamble
\usepackage{listings}
\usepackage{todonotes}
\usepackage{comment}




\newcommand{\htw}{\emph{Hunt the Wumpus }}

\title{Hakuna Matata: A Logic-Based Agent for the \htw Game}
\subtitle{Team White}
\author{Filippo~De~Bortoli \and Aneta~Koleva \and Lorenz~Leutgeb}
\institute{Free University of Bozen-Bolzano\\[2mm] \texttt{\{\href{mailto:filippo.debortoli@stud-inf.unibz.it}{filippo.debortoli},\href{mailto:aneta.koleva@stud-inf.unibz.it}{aneta.koleva},\href{mailto:lorenz.leutgeb@stud-inf.unibz.it}{lorenz.leutgeb}\}\newline @stud-inf.unibz.it}}
\date{01.06.2018}

\begin{document}

  \maketitle

  \begin{frame}{Outline}
    \tableofcontents
  \end{frame}

  \section{Introduction}

  \begin{frame}{Task}
    \begin{enumerate}
      \item Develop an intelligent agent that plays the \htw game.
      \item Implement a strategy to complete the knowledge about the state of the world and to be able to reason with respect to it. 
      \item Find an optimal strategy, i.e. maximize score.
    \end{enumerate}
  \end{frame}
%not sure if a slide with explanation of the game is needed

  \begin{frame}{Tools of the Trade}
  \begin{enumerate}
    %\item Pragmatic choices to avoid any confusions
    %\item Keeping it DRY (Don't Repeat Yourself)
    %\item Avoid reinvention
    \item Simulator rewritten in Python 3.6 (removes C dependency)
  \end{enumerate}

  \begin{center}
  \begin{tabular}{lll}
    \textbf{Tool/Library} & \textbf{Version} & \textbf{Purpose} \\
    Python & 3.6 & Runtime \\
    PyTest & 3.0 & Benchmarking \\
    DLV & Dec 2012 & ASP Solver \\
    networkX & 2.1 & Graph management \\
    ? & ? & Dependency management \\
  \end{tabular}
  \end{center}
  \end{frame}

  \section{Ideas and used concepts}

\begin{comment}
  \begin{frame}{Hybrid Agent}
  	
      \begin{tabular}{lll}
        \textbf{Component} & \textbf{Task} & \textbf{Language} \\
        Perceptor & Retrieves what is perceived in current location & Python \\
        Long-term Memory & Stores information about what has been discovered and done & Python \\
        Action Graph & Searches for optimal actions to be taken & Python \\
        Working Memory & Stores information needed to decide next action & Python \\
        Planner & Infers new knowledge and decides next action & ASP \\ 
      \end{tabular}
  
  \end{frame}
   
\end{comment}


 \begin{frame}{World Knowledge}
	
	  \begin{itemize}
		\item Static, partly-observable environment.
		\item Incomplete initial knowledge.
		\item Safety first!
		\item Discovers size of the world when it bumps. 
		\item Dangers and how to reason to avoid them. 
	  \end{itemize}
 \end{frame}

 \begin{frame}{Why ASP and DLV?}
	\begin{itemize}
		\item Previous experience with ASP.
		\item Allows to model non-monotonic reasoning through CWA. 
		\item Combines a high-level
		logic with grounding and solving. 
		\item DLV allows processing of incomplete knoledge. 
		\item At each time step, ASP solver called after perception has been caught.
		
	\end{itemize}
 \end{frame}

  \begin{frame}{$A^{\star}$ Search and World Exploration}
	 \begin{itemize}
	 	\item Build a reachability graph such that:
		 	\begin{itemize}
				 \item Vertices are all the safe and reachable rooms that still haven't been explored.
				 \item Edges are the connections between these rooms.
		    \end{itemize}
	    \item Calculate the weights using the cost function: 
	    $$
	    g(C,C^\prime) := M((X,Y),(X^\prime,Y^\prime)) + R(O,O^\prime)
	    $$
	    % not sure if we should add more about the cost funstion here
	    \item Compute the minimal cost of reaching the target room using \emph{$A^{\star}$ search} : 
	    \(f(n) = g(n) + h(n)\)
	    \item Deciding a goal cell depending on the current mode of the agent .
	    \item Choosing next cell depending on the goal one.
		 
	 \end{itemize}
  \end{frame}

\begin{frame}{Reachability graph example}
	\begin{figure}\centering
		\includegraphics{graph}
		\caption{Architecture }
	\end{figure}
\end{frame}


  \begin{frame}{Different modes}
   \begin{enumerate}
   	\item Explore - until the gold is discovered or there are more unexplored cells 
   	\item Grab - when glitter is perceived
   	\item Kill - not in any other mode and the gold is still up for grabs
		   	\begin{itemize}
		   		\item can try to kill the wumpus?
		   		\item should it try to kill the wumpus?
		   		\item can it shoot?
		   	\end{itemize}
   	\item Escape - not in any other mode
   \end{enumerate}
  \end{frame}

\iffalse
  \begin{frame}{ASP Encoding}
    \begin{center}
      \begin{tabular}{ll}
        \textbf{Predicate} & \textbf{Meaning} \\
        now/3 & position and orientation of the agent \\
        stench/2 & stench has been found in here \\
        wumpusDead/0 & a scream has been perceived \\
        grabbed/0 & glitter perceived, gold has been grabbed \\
      \end{tabular}
    \end{center}
  \end{frame}

  \begin{frame}{Heuristics}
    % TODO.
  \end{frame}
\fi
  \section{Implementation}
	%something more here maybe? it's a section with one slide (diagram)
  \begin{frame}{Architecture of the implementation}
  	\begin{figure}\centering
  		\includegraphics{architecture}
  		\caption{Architecture }
  	\end{figure}
	
  \end{frame}


  \section{Evaluation}
  
  \begin{frame}{Perfect agent}
    Performance of the agent compared against a \alert{perfect} agent.
    \begin{itemize}
      \item \alert{Omniscience}, i.e. total knowledge of the world 
      \item Optimal solution as shortest path over the action graph
      \item Score of this agent taken as reference for a dungeon
    \end{itemize}
  \end{frame}

  \begin{frame}{The Test Suite}
    %Powered by \alert{PyTest}.

  \end{frame}

  \section{Conclusion}

  \begin{frame}{Possible Improvements}
      % TODO: I think this can be skipped, if we do a good job.
  \end{frame}

  \begin{frame}{Conclusion}
  \end{frame}

  \begin{frame}[standout]
    Thank you!
  \end{frame}

\end{document}
