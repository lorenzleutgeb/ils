\hypertarget{hunt-the-wumpus}{%
\section{Hunt the Wumpus}\label{hunt-the-wumpus}}

This is an implementation of an agent that plays ``Hunt the Wumpus''.

Input should be given as follows:

\begin{verbatim}
#maxint = 1000.
\end{verbatim}

\hypertarget{constants}{%
\subsection{Constants}\label{constants}}

We use constants to encode concrete orientations, actions and modes, and
one corresponding predicate to all of them respectively.

According to \texttt{wumpus.common.Orientation} we define orientations.

\begin{verbatim}
#const right = 0.
#const up    = 1.
#const left  = 2.
#const down  = 3.

isOrientation 0..3
\end{verbatim}

According to \texttt{wumpus.common.Action} we define actions.

\begin{verbatim}
#const goforward = 0.
#const turnleft  = 1.
#const turnright = 2.
#const grab      = 3.
#const shoot     = 4.
#const climb     = 5.

isAction 0..5
\end{verbatim}

According to \texttt{wumpus.agent.Mode} we define modes.

\begin{verbatim}
#const explore = 0.
#const escape  = 1.
#const kill    = 2.

isMode 0..3
\end{verbatim}

Additionally we designate two orthogonal orientations as axes.

\begin{verbatim}
axis right
axis up
\end{verbatim}

Being able to mirror orientations/turns will come in handy.

\begin{verbatim}
mirror         left right
mirror         up   down
mirror         O2   O1
        mirror O1   O2
\end{verbatim}

\hypertarget{world-size}{%
\subsection{World Size}\label{world-size}}

If the agent never bumped against a wall, it has no information about
the size of the world.

\begin{verbatim}
sizeKnown
        bump _ _
\end{verbatim}

We will therefore derive the size of the world from something already
known.

\begin{verbatim}
exploredSize     X
        explored X Y

exploredSize       Y
        explored X Y
\end{verbatim}

Since any cell we already explored certainly is part of the world, we
can use this information and guess that the world is ``just a little''
bigger than we already know.

\begin{verbatim}
size                            S
        not sizeKnown
        #succ             Sm    S
        #max              Sm Se
                exploredSize Se
\end{verbatim}

Of course, if the agent bumped, the size can be directly inferred.

\begin{verbatim}
size               S
        bump  Sm Y
        >     Sm Y
        #prec Sm   S

size               S
        bump  X Sm
        <=    X Sm
        #prec   Sm S
\end{verbatim}

\hypertarget{cells-neighbors-facing-and-bumps}{%
\subsection{Cells, Neighbors, Facing And
Bumps}\label{cells-neighbors-facing-and-bumps}}

Span the space of cells.

\begin{verbatim}
cell         X Y
        #int X
        #int   Y
        >      Y 0
        >    X   0
        <=     Y S
        <=   X   S
        size     S

corner 1 1

corner       X 1
        cell X 1
        size X
        sizeKnown

corner       1 Y
        cell 1 Y
        size   Y
        sizeKnown

corner       S S
        cell S S
        size S
        sizeKnown
\end{verbatim}

Two cells are different if any of the components X, Y are unequal.

\begin{verbatim}
diff         X1 Y1 X2 Y2
        cell X1 Y1
        cell       X2 Y2
        !=   X1    X2

diff         X1 Y1 X2 Y2
        cell X1 Y1
        cell       X2 Y2
        !=      Y1    Y2
\end{verbatim}

\hypertarget{distances-and-costs}{%
\subsubsection{Distances and Costs}\label{distances-and-costs}}

The two predicates \texttt{outgoing} and \texttt{incoming} will help us
to keep the space for which we have to compute costs small.

\begin{verbatim}
neighborhood              X2 Y2
        now         X1 Y1       _
        anyNeighbor X1 Y1 X2 Y2
        safe              X2 Y2

outgoing             X Y
        neighborhood X Y

outgoing     X Y
        now  X Y _

incoming     X Y
        safe X Y
\end{verbatim}

We define distance along axes

\begin{verbatim}
distAlong        X1 Y1 X2 Y2 right D
        cell     X1 Y1
        cell           X2 Y2
        #absdiff       X1 X2       D

distAlong        X1 Y1 X2 Y2 up D
        #absdiff       Y1 Y2    D
        cell     X1 Y1
        cell           X2 Y2
\end{verbatim}

And using that, manhattan/taxicab distance is a charm.

\begin{verbatim}
manhattan            X1 Y1 X2 Y2             M
        outgoing     X1 Y1
        distAlong    X1 Y1 X2 Y2 right D1
        distAlong    X1 Y1 X2 Y2 up    D2
        +                              D1 D2 M

pathTurnCost          X Y O X Y 0
        outgoing  X Y
        isOrientation     O

pathTurnCost          X1 Y1 O1 X2 Y2 0
        isOrientation       O1
        facing        X1 Y1 _ X2 Y2
        diff          X1 Y1   X2 Y2

pathTurnCost             X1 Y1 O1 X2 Y2 1
        outgoing         X1 Y1
        isOrientation          O1
        incoming                  X2 Y2
        not pathTurnCost X1 Y1 O1 X2 Y2 0

departureTurnCost        X1 Y1 O1 X2 Y2    C
        outgoing     X1 Y1
        isOrientation          O1
        incoming                  X2 Y2
        #min                               C Cm
                reach    X1 Y1    X2 Y2 O2
                turnCost       O1       O2   Cm

rotCost                   X1 Y1 O1 X2 Y2       C
        outgoing          X1 Y1
        isOrientation           O1
        incoming                   X2 Y2
        pathTurnCost      X1 Y1 O1 X2 Y2 Cp
        departureTurnCost X1 Y1 O1 X2 Y2    Cd
        +                                Cp Cd C
\end{verbatim}

We use the Manhattan distance as cost function.

\begin{verbatim}
cost                      X1 Y1 O1 X2 Y2 C
        outgoing          X1 Y1
        isOrientation           O1
        incoming                   X2 Y2
        rotCost           X1 Y1 O1 X2 Y2 Cr
        manhattan         X1 Y1    X2 Y2    Cm
        +                                Cr Cm C
\end{verbatim}

The turn predicate associates two orientations with the action that must
be taken in order to turn from the first orientation to the second (if
possible).

\begin{verbatim}
turn up    left  turnleft
turn left  down  turnleft
turn down  right turnleft
turn right up    turnleft

turn up    right turnright
turn down  left  turnright
turn left  up    turnright
turn right down  turnright

turnCost              O O 0
        isOrientation O

turnCost                  O1 O2 1
            isOrientation O1
            isOrientation     O2
            !=            O1 O2
        not mirror        O1 O2

turnCost               O1 O2 2
        isOrientation  O1
        isOrientation     O2
        mirror         O1 O2
\end{verbatim}

Neighboring cells along the horizontal and vertical axis.

\begin{verbatim}
neighbor      X1 Y X2 Y right
        cell  X1 Y
        cell       X2 Y
        #succ X1   X2

neighbor      X Y1 X Y2 up
        cell  X Y1
        cell       X Y2
        #succ   Y1   Y2

neighbor         X1 Y1 X2 Y2 O2
        neighbor X2 Y2 X1 Y1 O1
        mirror               O1 O2

anyNeighbor           X1 Y1 X2 Y2
        neighbor      X1 Y1 X2 Y2 O
        isOrientation             O

twoNeighbors        X1 Y1 X2 Y2 X3 Y3
        anyNeighbor X1 Y1 X2 Y2
        anyNeighbor X1 Y1       X3 Y3
        diff              X2 Y2 X3 Y3

square               X1 Y1 X2 Y2 X3 Y3 X4 Y4
        twoNeighbors X1 Y1 X2 Y2 X3 Y3
        twoNeighbors X4 Y4 X2 Y2 X3 Y3
        diff         X1 Y1             X4 Y4

triple           X1 Y1 X2 Y2 X3 Y3
        neighbor X1 Y1 X2 Y2       A
        neighbor       X2 Y2 X3 Y3 A
        diff     X1 Y1       X3 Y3
        axis                       A
\end{verbatim}

Cells that agree on one component.

\begin{verbatim}
facing       X Z up X Y
        cell X Z
        <      Z      Y
        cell X        Y

facing       Z Y right X Y
        cell           X Y
        <    Z         X
        cell Z           Y

facing         X2 Y2    O2 X1 Y1
        facing X1 Y1 O1    X2 Y2
        mirror       O1 O2

reach        X1 Y1 X2 Y2 right
        cell X1 Y1
        cell       X2 Y2
        <    X1    X2

reach        X1 Y1 X2 Y2 up
        cell X1 Y1
        cell       X2 Y2
        <       Y1    Y2

reach        X1 Y1 X2 Y2 left
        cell X1 Y1
        cell       X2 Y2
        >    X1    X2

reach        X1 Y1 X2 Y2 down
        cell X1 Y1
        cell       X2 Y2
        >       Y1    Y2
\end{verbatim}

Complete information about information. We only do this here and not in
Python since we might have assumed a size.

\begin{verbatim}
-explored            X Y
        cell         X Y
        not explored X Y
\end{verbatim}

\hypertarget{do}{%
\subsection{Do}\label{do}}

\begin{verbatim}
priority
        shouldShoot

priority
        shouldClimb

priority
        shouldGrab

priority
        shouldAim

do shoot
        shouldShoot

do grab
        shouldGrab

do climb
        shouldClimb

do              A
        succAim A

do                   A
            towards  A
        not priority

shouldShoot
        mode   kill
        now    X Y O
        attack X Y O
\end{verbatim}

Pick gold if there's some in the cell, independently of the current
mode.

\begin{verbatim}
shouldGrab
        now     X Y O
        glitter X Y
\end{verbatim}

Climb if gold is picked and back to initial cell.

\begin{verbatim}
shouldClimb
        now 1 1 O
        mode escape
\end{verbatim}

Generally we will be turning and go forward towards a goal. Exceptions
are when we want to take high priority actions like grabbing, climbing
or shooting.

\begin{verbatim}
canFace                          A
        facing X1 Y1    O2 X2 Y2
        now    X1 Y1 O1
        next               X2 Y2
        !=           O1 O2
        turn         O1 O2       A

-cannotFace
        isAction A
        canFace  A

cannotFace
        not -cannotFace

towards goforward
        facing X1 Y1 O1 X2 Y2
        now    X1 Y1 O1
        next            X2 Y2

towards         A
        canFace A
        not towards goforward
\end{verbatim}

TODO: Can we turn in a better way?

\begin{verbatim}
towards turnleft
        not towards goforward
        cannotFace
\end{verbatim}

\hypertarget{detection-of-wumpus}{%
\subsection{Detection Of Wumpus}\label{detection-of-wumpus}}

\begin{verbatim}
wumpus         X1 Y1
        square X1 Y1 X2 Y2 X3 Y3 X4 Y4
        stench       X2 Y2
        stench             X3 Y3
        explored                 X4 Y4
        -wumpusDead
\end{verbatim}

Antipodal matching.

\begin{verbatim}
wumpus               X2 Y2
        triple X1 Y1 X2 Y2 X3 Y3
        stench X1 Y1
        stench             X3 Y3
        -wumpusDead
\end{verbatim}

Corner case.

\begin{verbatim}
wumpus                     XW YW
        twoNeighbors XC YC XW YW XE YE
        stench       XC YC
        corner       XC YC
        explored                 XE YE
        -wumpusDead
\end{verbatim}

Auxiliary flag to signal detection of wumpus.

\begin{verbatim}
wumpusDetected
        cell  X Y
        wumpus X Y
\end{verbatim}

If wumpus is not certainly detected, we must exclude other cells where
it might be.

\begin{verbatim}
possibleWumpus            X2 Y2
        anyNeighbor X1 Y1 X2 Y2
        stench      X1 Y1
        -explored         X2 Y2
        not shotAt        X2 Y2
        -wumpusDead
        not wumpusDetected

shotAt                  X Y
        facing XS YS OS X Y
        shot   XS YS OS
\end{verbatim}

\hypertarget{detection-of-pits}{%
\subsection{Detection Of Pits}\label{detection-of-pits}}

A neighbor of a breeze is a possible pit if it can be.

\begin{verbatim}
possiblePit         XB YB XP YP
        breeze      XB YB
        anyNeighbor XB YB XP YP
        not -pit          XP YP
\end{verbatim}

Any explored cell certainly cannot be a pit, otherwise we would be dead
by now.

\begin{verbatim}
-pit             X Y
        explored X Y
\end{verbatim}

If we have explored a cell already and felt no brezee, then no neighbor
can be a pit.

\begin{verbatim}
-pit                      X2 Y2
        anyNeighbor X1 Y1 X2 Y2
        -breeze     X1 Y1
\end{verbatim}

Finally, we assume that there are pits where there might be possible
pits.

\begin{verbatim}
pit                     X Y
        possiblePit _ _ X Y
\end{verbatim}

\hypertarget{safety-of-cells}{%
\subsection{Safety Of Cells}\label{safety-of-cells}}

Any cell where the wumpus is is not safe.

\begin{verbatim}
safe             X Y
        explored X Y

-safe          X Y
        wumpus X Y

-safe                  X Y
        possibleWumpus X Y

-safe       X Y
        pit X Y

safe              X Y
        reachable X Y
        not -safe X Y
\end{verbatim}

\hypertarget{explore-mode}{%
\subsection{Explore Mode}\label{explore-mode}}

\begin{verbatim}
reachable           X Y
        anyNeighbor X Y XE YE
        explored        XE YE

reachable        X Y
        explored X Y
\end{verbatim}

Auxiliary flag to signal whether we can still explore further.

\begin{verbatim}
frontier          X Y
        reachable X Y
        safe      X Y
        -explored X Y

shouldExplore
        frontier X Y
\end{verbatim}

\hypertarget{kill-mode}{%
\subsection{Kill Mode}\label{kill-mode}}

\begin{verbatim}
dontShoot
        possibleWumpus X Y
        pit            X Y

dontShoot
        wumpus X Y
        pit    X Y

dontShoot
        not haveArrow

dontShoot
        wumpusDead

dontShoot
        grabbed

canKill        XS YS OS
        facing XS YS OS XW YW
        wumpus          XW YW
        safe   XS YS

shouldKill
        canKill _ _ _
        not dontShoot

canTryKill       X Y O
        facing   X Y O XC YC
        safe     X Y
        possibleWumpus XC YC

shouldTryKill
            canTryKill _ _ _
        not shouldKill
        not dontShoot

attack             X Y O
        canTryKill X Y O
        shouldTryKill

attack          X Y O
        canKill X Y O
        shouldKill

shouldAttack
        attack _ _ _

aim                        OS
            !=           O OS
            attack XS YS   OS
        not attack XS YS O
            now    XS YS O

canAim                A
        turn    O1 O2 A
        aim        O2
        now _ _ O1

-cannotAim
        canAim _

cannotAim
        not -cannotAim

succAim        A
        canAim A
        mode   kill

succAim turnleft
        aim _
        cannotAim
        mode kill

shouldAim
        succAim A
\end{verbatim}

\hypertarget{optimization-for-goal}{%
\subsection{Optimization For Goal}\label{optimization-for-goal}}

\begin{verbatim}
h                     X2 Y2 C
        cost X1 Y1 O1 X2 Y2 C
        now  X1 Y1 O1
\end{verbatim}

Interesting candidates are those cells that we have not yet explored and
we know that they are safe.

\begin{verbatim}
candidate 1      X Y C
        h        X Y C
        frontier X Y
        mode     explore

candidate 1    X Y C
        h      X Y C
        attack X Y _
        mode   kill

candidate 1 1 1 C
        h   1 1 C
        mode escape

candidate 2                     X Y         C
        now         XN YN ON
        anyNeighbor XN YN   X Y
        safe                X Y
        goal                      XG YG
        cost                X Y O XG YG C2
        reach       XN YN X Y O
        + C1 C2 C
        + Cr 1 C1
        departureTurnCost XN YN ON X Y Cr
        not easyGoal

candidate      2 X Y C
        choice 1 X Y C
        easyGoal

level 1
level 2

next X Y
        choice 2 X Y C

goal X Y
        choice 1 X Y C

easyGoal
        now         XN YN _
        anyNeighbor XN YN XG YG
        goal              XG YG

foundNext
        next _ _
\end{verbatim}

TODO: Inconsistent? - foundGoal not foundNext

\begin{verbatim}
foundGoal
        goal _ _
\end{verbatim}

TODO: Inconsistent? - not foundGoal not priority

\begin{verbatim}
choice L X Y C | -choice L X Y C
        level     L
        candidate L X Y C

~
        level     L
        candidate L X1 Y1 C1
        candidate L          X2 Y2 C2
        choice    L          X2 Y2 C2
        <=                C1       C2
\end{verbatim}

\hypertarget{autopilot}{%
\subsection{Autopilot}\label{autopilot}}

\begin{verbatim}
-autopilot
        not autopilot

autopilot
        not priority
        mode explore

autopilot
        not priority
        mode escape
\end{verbatim}

\hypertarget{mode-selector}{%
\subsection{Mode Selector}\label{mode-selector}}

\begin{verbatim}
mode grab
        shouldGrab

mode explore
         not mode grab
        shouldExplore
        -grabbed

mode kill
        not mode grab
        not mode explore
        shouldAttack
        -grabbed

mode escape
        not mode grab
        not mode explore
        not mode kill
\end{verbatim}